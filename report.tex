%%%%%%%%%%%%%%%%%%%%%%%%%%%%%%%%%%%%%%%%%
% Large Colored Title Article
% LaTeX Template
% Version 1.1 (25/11/12)
%
% This template has been downloaded from:
% http://www.LaTeXTemplates.com
%
% Original author:
% Frits Wenneker (http://www.howtotex.com)
%
% License:
% CC BY-NC-SA 3.0 (http://creativecommons.org/licenses/by-nc-sa/3.0/)
%
%%%%%%%%%%%%%%%%%%%%%%%%%%%%%%%%%%%%%%%%%

%----------------------------------------------------------------------------------------
%	PACKAGES AND OTHER DOCUMENT CONFIGURATIONS
%----------------------------------------------------------------------------------------

\documentclass[DIV=calc, paper=a4, fontsize=12pt, twocolumn]{article}	 % A4 paper and 11pt font size

\usepackage{graphicx} % enables importing images
\usepackage[cm]{fullpage} % page margin fix
\usepackage[english]{babel} % English language/hyphenation
\usepackage[protrusion=true,expansion=true]{microtype} % Better typography
\usepackage{amsmath,amsfonts,amsthm} % Math packages
\usepackage[svgnames]{xcolor} % Enabling colors by their 'svgnames'
\usepackage[hang, small,labelfont=bf,up,textfont=it,up]{caption} % Custom captions under/above floats in tables or figures
\usepackage{booktabs} % Horizontal rules in tables
\usepackage{fix-cm}	 % Custom font sizes - used for the initial letter in the document
\usepackage{float}

\usepackage{sectsty} % Enables custom section titles
\usepackage{titlesec} % enables inserting margins around sections and subsections

\allsectionsfont{\usefont{OT1}{phv}{b}{n}} % Change the font of all section commands

\usepackage{pgfplots} % package to plot graphs using data
\usepackage{fancyhdr} % Needed to define custom headers/footers
\pagestyle{fancy} % Enables the custom headers/footers
\usepackage{lastpage} % Used to determine the number of pages in the document (for "Page X of Total")

% Headers
% raisebox moves text in header up, header text needs to be enclosed in curly brackets
\lhead{\raisebox{0.5\height} {COMS30127 - Computational Neuroscience}}
\chead{}
\rhead{\raisebox{0.5\height} {mm13354}}

% Footers
\lfoot{}
\cfoot{}
\rfoot{\footnotesize Page \thepage\ of \pageref{LastPage}} % "Page 1 of 2"

\renewcommand{\headrulewidth}{0.2pt} % Thin header rule
\renewcommand{\footrulewidth}{0.4pt} % Thin footer rule
\headsep 5pt
\usepackage{lettrine} % Package to accentuate the first letter of the text
\newcommand{\initial}[1]{ % Defines the command and style for the first letter
	\lettrine[lines=3,lhang=0.3,nindent=0em]{
		\color{Teal}
		{\textsf{#1}}}{}}

\setlength{\intextsep}{2pt}
\setlength{\textfloatsep}{2pt}
%----------------------------------------------------------------------------------------
%	TITLE SECTION
%----------------------------------------------------------------------------------------

\usepackage{titling} % Allows custom title configuration

\newcommand{\HorRule}{\color{Teal} \rule{\linewidth}{1pt}} % Defines the blue horizontal rule under the title
	

\pretitle{\vspace{-10pt} \begin{center} \fontsize{25}{25} \usefont{OT1}{phv}{b}{n} \color{Teal} \selectfont} % Horizontal rule before the title
	
	\title{Neuron Model} % Your article title
	
	\posttitle{\end{center}} % Whitespace under the title

\preauthor{ \begin{center}\large \lineskip -1em \usefont{OT1}{phv}{b}{sl} \color{Teal}} % Author font configuration
	
	\author{Maria Marinova} % Your name
	
	\postauthor{\vspace{-10pt} \footnotesize \usefont{OT1}{phv}{m}{sl} \color{Black} % Configuration for the institution name
		
		\par\end{center}\HorRule } % Horizontal rule after the title

\date{\vspace{-30pt}} % Add a date here if you would like one to appear underneath the title block

%----------------------------------------------------------------------------------------

\begin{document}
	
	\maketitle % Print the title
	
	\thispagestyle{fancy} % Enabling the custom headers/footers for the first page 
	
	%----------------------------------------------------------------------------------------
	%	INTRODUCTION
	%----------------------------------------------------------------------------------------
	
	% The first character should be within \initial{}
	\initial{T}\textbf{he aim of this report is to highlight different cases of neuron modelling. The analysis is largely based on graphs, and the language of implementation is Python.}
	
	%----------------------------------------------------------------------------------------
	%	ARTICLE CONTENTS
	%----------------------------------------------------------------------------------------
	\vfill
	\subsection*{Q1. Integrate and fire model}
		\begin{figure}[H]
			\includegraphics*[width = 9.5cm]{q1}
		\end{figure}
	
	A neuron simulation with the integrate and fire model, Euler's method is used with a timestep $\delta t = 1 ms$. The current is enough for multiple spikes to be produced.
	\vfill
		
	\subsection*{Q2. Minimum current for action potential}
		\begin{equation}
			I_e = \dfrac{V_t - E_l}{R_m}
		\end{equation}
		\begin{equation}
			I_e = \dfrac{-40 mV - (-70 mV)}{10M\Omega}
		\end{equation}
		\begin{equation}
			I_e = \dfrac{30 mV}{10M\Omega} = 3.0 nA
		\end{equation}
	\vfill
		
	\subsection*{Q3. Integrate and fire model with lower than the minimum current}
	The current is $0.1 nA$ lower that the current needed for a spike -- $I_e = 2.9 nA$ -- hence no spikes are observed.
		\begin{figure}[H]
			\includegraphics*[width = 9.5cm]{q3}
		\end{figure}
	\vfill	
		
	\subsection*{Q4. Firing rate as function of the input current}
	Plot of the fire rate as a function of the input current. Spikes start being produced at $I_e = 3.0 nA$, which is the minimum input current for action potential.
		\begin{figure}[H]
			\includegraphics*[width = 9.5cm]{q4}
		\end{figure}
		
	\vfill
	\subsection*{Q5. Interprojection of two neurons}
	In both cases two neurons with synaptic connections between each other and with the same parameters are simulated. 
	\par 
	In the first case, the synapses are excitatory with reversal potential $E_s = 0 mV$. It can be observed that the two neurons' membrane potential converges.
		\begin{figure}[H]
			\includegraphics*[width = 9.5cm]{q5a}
		\end{figure}
	In the second case, the synapses are inhibitory with reversal potential $E_s = -80 mV$. Unlike in the previous case, the two neurons' membrane potential diverges.
		\begin{figure}[H]
			\includegraphics*[width = 9.5cm]{q5b}
		\end{figure}
	\vfill
	\eject
	\subsection*{Q6. Integrate and fire model with potassium current}
	The neuron simulated in this question has the same parameters as the one in Q1. However, a slow potassium current with reversal potential $E_K = -80mV$ is added.
	\par
		\begin{figure}[htb]
			\includegraphics*[width = 9.5cm]{q6} 
		\end{figure}
	
	%------------------------------------------------
	
\end{document}